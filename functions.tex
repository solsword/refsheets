\documentclass{handout}
\lstset{language=Python}

\title{Functions}

\cfoot{\emph{Note: To fit examples on this page, short and meaningless variable and function names have been used. \textbf{DO NOT} do this in your own code.}}

\begin{document}
\begin{multicols*}{2}

\section{Syntax}

A function definition has four parts:
\begin{enumerate}
\item A \term{name}
\item Zero or more \term{parameters}
\item At least one \term{body} statement
\item Optionally, a \code{return} statement
\end{enumerate}

\begin{minipage}[t]{0.42\columnwidth}
Without parameters:
\begin{lstlisting}
def f():
    print("Hello")
\end{lstlisting}
\end{minipage}
\begin{minipage}[t]{0.05\columnwidth}
\ 
\end{minipage}
\begin{minipage}[t]{0.42\columnwidth}
Parameters + return:
\begin{lstlisting}
def g(x, y):
    return x+y
\end{lstlisting}
\end{minipage}

\noindent
\textsl{In these examples \code{f} and \code{g} are the \term{names}, and on the right \code{x} and \code{y} are the \term{parameters}.}

\section{Usage}

\term{Call} a function by itself or as part of an \term{expression} by writing its \term{name} and putting \term{parameters} in parentheses afterwards:

\begin{lstlisting}
f() # prints "Hello"
print(5*g(2, 3)) # g returns 5, so this prints 25
\end{lstlisting}

Instead of copying and pasting code, you can use a function:

\begin{minipage}[t]{0.42\columnwidth}
\begin{lstlisting}
animal = "dog"
print(
    "I love",
    animal + "s"
)

name = "cat"
print(
    "I love",
    animal + "s"
)
\end{lstlisting}
\end{minipage}
\begin{minipage}[t]{0.05\columnwidth}
\  \\
\hspace*{1ex}$\rightarrow$
\end{minipage}
\begin{minipage}[t]{0.42\columnwidth}
\begin{lstlisting}
def show_love(animal):
    print(
        "I love",
        animal + "s"
    )

show_love("dog")
show_love("cat")
\end{lstlisting}
\end{minipage}

\noindent
\term{Parameters} work just like normal \term{variables}, but they get their values from the \term{function call}. The same \term{parameter} may have different values if a function is \term{called} multiple times. Values supplied as \term{parameters} are called \term{arguments}.

\textsl{In the example above, the \term{parameter} \code{animal} has the value
\code{"dog"} during the first \term{function call} to \code{show_love}, and the
value \code{"cat"} during the second \term{function call}.}


\end{multicols*}
\end{document}
